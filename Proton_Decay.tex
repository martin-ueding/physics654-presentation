% Copyright © 2015-2016 Martin Ueding <dev@martin-ueding.de>
%
\documentclass[english, fleqn]{beamer}

\usetheme{Malmoe}

\usepackage[bibatend]{header}

\DeclareSIUnit\year{yr}


\renewcommand\iup{\text i}
\renewcommand\eup{\text e}

\title{Proton Decay}
%\subtitle{}
\author{Martin Ueding -- mu@martin-ueding.de}
%\date{}

\begin{document}

\begin{frame}
    \titlepage
\end{frame}

\section{Standard Model}

\section{Motivation}

\subsection{Matter--antimatter asymmetry}

\section{Models}

\subsection{Georgi-Glashow model}

\subsection{Pati-Salam}

\begin{frame}
    \parencite{Wu/Proton_decay}
\end{frame}

\section{Experimental evidence}

\subsection{Quick estimation}

\begin{frame}
    \begin{block}{Observation}
        Our bodies are not very radioactive.
    \end{block}

    \pause

    \begin{block}{Proton lifetime}
        \begin{itemize}
            \item We would notice \SI{1}{\sievert\per\year}
            \item Body consists half of protons
            \item Proton lifetime greater than \SI{7e23}{\second} or \SI{2e16}{\year}
        \end{itemize}
    \end{block}

    \nocite{wikipedia/groessenordnung-aequivalentdosis}
\end{frame}

\subsection{Super Kamiokande}

\begin{frame}
    References

    \printbibliography
\end{frame}

\end{document}

% vim: spell spelllang=en_us
